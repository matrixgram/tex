\documentclass{beamer}
\usepackage{xepersian} % Load the xepersian package for Persian support

% Set fonts for XePersian
\settextfont{Yas} % Choose a Persian font such as Yas, IranNastaliq, etc.
\setlatintextfont{Times New Roman}

% Choose a Beamer theme
\usetheme{CambridgeUS}

% Custom Color Theme
\usecolortheme{dove}
\setbeamercolor{structure}{fg=darkblue} % Change the main theme color
\setbeamercolor{frametitle}{bg=darkred, fg=white} % Customize frame title color
\setbeamercolor{block title}{bg=orange!85!black, fg=white} % Customize block titles
\setbeamercolor{block body}{bg=orange!10!white} % Customize block body background

% Title Information
\title[نمونه ارائه]{نمونه‌ای از ارائه بیمر با XePersian}
\author{نام شما}
\institute{موسسه شما}
\date{\today}

\begin{document}
	
	% Title Slide
	\begin{frame}
		\titlepage
	\end{frame}
	
	% Outline Slide
	\begin{frame}{فهرست مطالب}
		\tableofcontents
	\end{frame}
	
	% Section 1
	\section{مقدمه}
	\begin{frame}{مقدمه‌ای بر استفاده از XePersian}
		\begin{itemize}
			\item بسته XePersian از XeLaTeX برای پشتیبانی از زبان‌های راست به چپ استفاده می‌کند.
			\item این بسته قابلیت‌های مناسبی برای کار با زبان فارسی دارد.
			\item همچنین با ارائه‌های بیمر به خوبی سازگار است.
		\end{itemize}
	\end{frame}
	
	% Section 2
	\section{ویژگی‌های بیمر}
	\begin{frame}{ویژگی‌های بسته بیمر}
		\begin{block}{چرا بیمر؟}
			\begin{itemize}
				\item مناسب برای ارائه‌های علمی
				\item پشتیبانی از قالب‌های مختلف
				\item قابلیت تغییر رنگ، فونت، و ساختار
			\end{itemize}
		\end{block}
		
		\begin{alertblock}{توجه}
			برای استفاده از فونت‌های فارسی، حتماً از XeLaTeX استفاده کنید.
		\end{alertblock}
	\end{frame}
	
	% Section 3
	\section{نتیجه‌گیری}
	\begin{frame}{نتیجه‌گیری}
		\begin{itemize}
			\item بیمر ابزاری قدرتمند و انعطاف‌پذیر است.
			\item بسته XePersian امکان ارائه‌های فارسی را فراهم می‌کند.
		\end{itemize}
		\bigskip
		\centering
		از توجه شما متشکرم!
	\end{frame}
	
\end{document}
