\documentclass{article}
\usepackage{amsmath}
\usepackage{amsfonts}

\begin{document}
	
	\title{Proof of Fibonacci Computation Using Matrix Exponentiation}
	\author{Mohhamad hayEri}
	\date{10/14/24}
	\maketitle
	
	\section*{Introduction}
	The Fibonacci sequence can be computed using matrix exponentiation, achieving a time complexity of $O(\log n)$. We will prove this by following these steps:
	
	1. Matrix representation of Fibonacci numbers.
	2. Matrix multiplication.
	3. Matrix exponentiation.
	4. Final result.
	
	\section*{Step 1: Matrix Representation of Fibonacci Numbers}
	The Fibonacci numbers can be expressed in terms of matrix multiplication as follows:
	
	$$\begin{bmatrix} F(n) \\ F(n-1) \end{bmatrix} = \begin{bmatrix} 1 & 1 \\ 1 & 0 \end{bmatrix}^{n-1} \begin{bmatrix} F(1) \\ F(0) \end{bmatrix}$$
	
	Where:
	$$F(0) = 0, \quad F(1) = 1$$
	
	The transformation matrix $M$ is defined as:
	
	$$M = \begin{bmatrix} 1 & 1 \\ 1 & 0 \end{bmatrix}$$
	
	\section*{Step 2: Matrix Multiplication}
	To multiply two $2 \times 2$ matrices $A$ and $B$:
	
	$$A = \begin{bmatrix} a & b \\ c & d \end{bmatrix}, \quad B = \begin{bmatrix} e & f \\ g & h \end{bmatrix}$$
	
	The product $C = A \cdot B$ is given by:
	
	$$C = \begin{bmatrix} ae + bg & af + bh \\ ce + dg & cf + dh \end{bmatrix}$$
	
	This multiplication takes constant time, $O(1)$, since it involves a fixed number of operations.
	
	\section*{Step 3: Matrix Exponentiation}
	To compute $M^k$ efficiently, we use the method of exponentiation by squaring:
	
	\begin{itemize}
		\item If $k = 0$, $M^0 = I$ (the identity matrix).
		\item If $k = 1$, $M^1 = M$.
		\item If $k$ is even, $M^k = M^{k/2} \cdot M^{k/2}$.
		\item If $k$ is odd, $M^k = M \cdot M^{k-1}$.
	\end{itemize}
	
	This method reduces the number of multiplications needed to compute $M^k$ to $O(\log k)$. Each multiplication of two $2 \times 2$ matrices takes $O(1)$, so the overall time complexity for matrix exponentiation is $O(\log n)$.
	
	\section*{Step 4: Final Result}
	To find $F(n)$:
	
	1. Compute $M^{(n-1)}$ using the matrix exponentiation method.
	2. The top left element of the resulting matrix $M^{(n-1)}$ will be $F(n)$.
	
	Thus, we have shown that the $n$-th Fibonacci number can be computed in $O(\log n)$ time using matrix exponentiation.
	
	\section*{Conclusion}
	The proof is complete, and we have established that the Fibonacci sequence can be computed efficiently using matrix exponentiation, achieving a time complexity of $O(\log n)$.
	
\end{document}