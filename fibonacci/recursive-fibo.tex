\documentclass{article}
\usepackage{amsmath}
\usepackage{amsfonts}
\usepackage{amssymb}

\begin{document}
	
	\title{Proof of Time Complexity for the Recursive Fibonacci Algorithm}
	\author{Mohammad Hayeri}
	\date{10/14/24}
	\maketitle
	
	\section*{Fibonacci Recursive Function}
	
	The naive recursive Fibonacci function can be defined as follows:
	
	\begin{verbatim}
		def fibonacci(n):
		    if n <= 1:
				return n
			return fibonacci(n - 1) + fibonacci(n - 2)
	\end{verbatim}
	
	\section*{Recurrence Relation}
	
	The time complexity $ T(n) $ of the Fibonacci function can be expressed as a recurrence relation:
	
	\begin{align*}
		\text{For } n = 0 \text{ or } n = 1: & \quad T(n) = O(1) \\
		\text{For } n > 1: & \quad T(n) = T(n - 1) + T(n - 2) + O(1)
	\end{align*}
	
	The $ O(1) $ term accounts for the constant time taken to perform the addition and the function call overhead.
	
	\section*{Simplifying the Recurrence Relation}
	
	Ignoring the constant term for simplicity, we can write:
	
	$$T(n) = T(n - 1) + T(n - 2)$$
	
	This recurrence relation is similar to the Fibonacci sequence itself. To solve this, we can use the characteristic equation method.
	
	\section*{Characteristic Equation}
	
	The characteristic equation for the recurrence relation $ T(n) = T(n - 1) + T(n - 2) $ can be derived as follows:
	
	1. Assume a solution of the form $ T(n) = r^n $.
	2. Substitute into the recurrence relation:
	
	$$r^n = r^{n-1} + r^{n-2}$$
	
	3. Dividing through by $ r^{n-2} $ (assuming $ r \neq 0 $):
	
	$$r^2 = r + 1$$
	
	4. Rearranging gives us the characteristic equation:
	
	$$r^2 - r - 1 = 0$$
	
	\section*{Solving the Characteristic Equation}
	
	Using the quadratic formula $ r = \frac{-b \pm \sqrt{b^2 - 4ac}}{2a} $:
	
	Here, $ a = 1, b = -1, c = -1 $:
	
	$$r = \frac{1 \pm \sqrt{5}}{2}$$
	
	The two roots are:
	
	$$r_1 = \frac{1 + \sqrt{5}}{2} \quad (\text{the golden ratio, approximately } 1.618)$$
	$$r_2 = \frac{1 - \sqrt{5}}{2} \quad (\text{approximately } -0.618)$$
	
	\section*{General Solution}
	
	The general solution to the recurrence relation can be expressed as:
	
	$$T(n) = A r_1^n + B r_2^n$$
	
	where $ A $ and $ B $ are constants determined by the initial conditions.
	
	\section*{Asymptotic Behavior}
	
	As $ n $ grows large, the term involving $ r_2 $ (which is negative and less than 1 in absolute value) becomes negligible. Therefore, the dominant term is:
	
	$$T(n) \approx A r_1^n$$
	
	Since $ r_1 $ is approximately $ 1.618 $, we can conclude that:
	
	$$T(n) = O(r_1^n) = O\left(\left(\frac{1 + \sqrt{5}}{2}\right)^n\right)$$
	
	\section*{Conclusion}
	
	The time complexity of the naive recursive Fibonacci algorithm can be expressed as:
	
	$$\text{Time Complexity: } O(2^n) \quad (\text{since } r_1 \approx 1.618 < 2, \text{ but grows exponentially})$$
	
	This analysis shows that the naive recursive Fibonacci algorithm has exponential time complexity due to the nature of the recursive calls, which can be modeled using a recurrence relation and solved using characteristic equations.
	
\end{document}